\documentclass[11pt, oneside]{article} 
\usepackage{geometry}
\geometry{letterpaper}
\usepackage{graphicx}	
\usepackage{amssymb}
\usepackage{amsmath}
\usepackage{natbib}
\usepackage{hyperref}
\hypersetup{colorlinks=true,linkcolor=blue,citecolor=blue,urlcolor=blue}
\usepackage{parskip}

\newcommand{\R}{\mathbb{R}}
\newcommand{\C}{\mathbb{C}}
\title{CSS-only quadrant rotations}
\author{}
\date{}

\begin{document}
\maketitle

\section{The first rotation}

We want to rotate a quadrant square of side length $L$ by $\pi/2$, translating it diagonally
as we go to avoid intersecting the crossbars.
As we do the rotation, the corner of the quadrant moves through an angle of $\pi/2$ from $-\pi/4$ to $\pi/4$,
requiring a shift of
\begin{align*}
m(t) &= \frac{L}{\sqrt{2}} \cos(\pi t/2 - \pi/4) d - \frac{L}{2} d = \cos(\pi t/2 - \pi/4) w + w_0
\end{align*}
as $t$ goes from 0 to 1, where $d = (\pm 1, \pm 1)$ is the appropriate diagonal vector.
In terms of the complex plain, the effect we want is the animated transform
\begin{align*}
f(t,z) &= w_0 + \cos(\pi t/2 - \pi/4) w + e^{i \pi t/2} z
\end{align*}
where $z \in \C$, $w \in \C$ is the maximum translation, and $t \in [0,1]$.  Let
\begin{align*}
e(t) &= \exp(2 \pi i t) \\
c(t) &= \cos(2 \pi t) \\
s(t) &= \sin(2 \pi t)
\end{align*}
so that
\begin{align*}
s(t) &= \frac{i}{2} (e(-t) - e(t)) \\
c(t) &= \frac{1}{2} (e(t) + e(-t))
\end{align*}
and $f(t,z)$ becomes
\begin{align*}
f(t,z) &= w_0 + c(t/4 - 1/8) w + e(t/4) z \\
  &= w_0 + (e(t/4 - 1/8) + e(-t/4 + 1/8)) w/2 + e(t/4) z \\
  &= w_0 + e(t/4)e(-1/8) w/2 + e(-t/4)e(1/8) w/2 + e(t/4) z \\
  &= w_0 + e(t/4)(e(-1/8) w/2 + e(-t/2)e(1/8) w/2 + z) \\
  &= w_0 + e(t/4)(e(-1/8) w/2 + e(-t/2) (e(1/8) w/2 + e(t/2) z))
\end{align*}
which is now expressible as primitive transforms:
\begin{align*}
f(t) &= T(w_0) R(t/4) T(e(-1/8) w/2) R(-t/2) T(e(1/8) w/2) R(t/2)
\end{align*}
where $R$, $T$ are rotate, translate.  If we set the CSS transform property to that sequence of 5 primitive
transforms, everything will animate correctly.

\section{The second rotation}

But wait!  If we extrapolate this into the next $\pi/2$, our translation will go in the wrong way.  To fix this, we
need to make our transition correction ping-pong back between $-\pi/4$ and $\pi/4$ without leaving that
interval:
\begin{align*}
f(t,z) &= w_0 + \cos(\pi/2 \operatorname{jag}(t) - \pi/4) w + e^{i \pi t/2} z \\
\operatorname{jag}(t) &= \min(\operatorname{rem}(t, 2), \operatorname{rem}(2-t,2))
\end{align*}
where $\operatorname{rem}$ is the (correctly signed) floating point remainder function.  Fortunately for us we
only need to evaluate $\operatorname{jag}$ at integers, where it has the simpler form
\begin{align*}
\operatorname{jag}(n) &= \operatorname{rem}(n, 2)
\end{align*}
Setting $j = \operatorname{jag}(t)$, we can repeat our derivation:
\begin{align*}
f(t,z) &= w_0 + c(j/4 - 1/8) w + e(t/4) z \\
  &= w_0 + (e(j/4 - 1/8) + e(-j/4 + 1/8)) w/2 + e(t/4) z \\
  &= w_0 + e(j/4)e(-1/8) w/2 + e(-j/4)e(1/8) w/2 + e(t/4) z \\
  &= w_0 + e(j/4)(e(-1/8) w/2 + e(-j/2)e(1/8) w/2 + e(t/4-j/4) z) \\
  &= w_0 + e(j/4)(e(-1/8) w/2 + e(-j/2) (e(1/8) w/2 + e(t/4+j/4) z)) \\
f(t) &= T(w_0) R(j/4) T(e(-1/8) w/2) R(-j/2) T(e(1/8) w/2) R(t/4+j/4)
\end{align*}

\end{document}  